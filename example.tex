
\extrapart{例} % -*- Mode: Lisp; Package: SCHEME; Syntax: Common-lisp -*-

\nobreak
手続き {\cf integrate-system} は微分方程式の系
$$y_k^\prime = f_k(y_1, y_2, \ldots, y_n), \; k = 1, \ldots, n$$
を Runge-Kutta 法で積分する。

パラメタ {\tt system-derivative} は系の状態 (状態変数 $y_1, \ldots, 
y_n$ に対する値からなるベクタ) を受け取って系の
微分係数 (値 $y_1^\prime, \ldots, y_n^\prime$) を算出する関数である。
パラメタ {\tt initial-state} は系の初期状態を定める。
そして {\tt h} は積分の刻み幅の初期推定値である。

{\cf integrate-system} が返す値は,系の状態からなる無限ストリームである。

\begin{schemenoindent}
(define (integrate-system system-derivative
                          initial-state
                          h)
  (let ((next (runge-kutta-4 system-derivative h)))
    (letrec ((states
              (cons initial-state
                    (delay (map-streams next
                                        states)))))
      states)))%
\end{schemenoindent}

手続き {\cf runge-kutta-4} は,系の状態から系の微分係数を算出する関数 {\tt f} を
受け取る。
これは,系の状態を受け取って新しい系の状態を算出する関数を
算出する。

\begin{schemenoindent}
(define (runge-kutta-4 f h)
  (let ((*h (scale-vector h))
        (*2 (scale-vector 2))
        (*1/2 (scale-vector (/ 1 2)))
        (*1/6 (scale-vector (/ 1 6))))
    (lambda (y)
      ;; y is a system state
      (let* ((k0 (*h (f y)))
             (k1 (*h (f (add-vectors y (*1/2 k0)))))
             (k2 (*h (f (add-vectors y (*1/2 k1)))))
             (k3 (*h (f (add-vectors y k2)))))
        (add-vectors y
          (*1/6 (add-vectors k0
                             (*2 k1)
                             (*2 k2)
                             k3)))))))

(define (elementwise f)
  (lambda vectors
    (generate-vector
     (vector-length (car vectors))
     (lambda (i)
       (apply f
              (map (lambda (v) (vector-ref  v i))
                   vectors))))))

(define (generate-vector size proc)
  (let ((ans (make-vector size)))
    (letrec ((loop
              (lambda (i)
                (cond ((= i size) ans)
                      (else
                       (vector-set! ans i (proc i))
                       (loop (+ i 1)))))))
      (loop 0))))

(define add-vectors (elementwise +))

(define (scale-vector s)
  (elementwise (lambda (x) (* x s))))%
\end{schemenoindent}

{\cf map-streams} 手続きは {\cf map} に相当する: これはその第一引数 (手続き) を
第二引数 (ストリーム) のすべての要素に適用する。

\begin{schemenoindent}
(define (map-streams f s)
  (cons (f (head s))
        (delay (map-streams f (tail s)))))%
\end{schemenoindent}

無限ストリームは,そのストリームの最初の要素を car が保持し,
そしてそのストリームの残りをかなえる約束を cdr が保持するペアとして
実装される。

\begin{schemenoindent}
(define head car)
(define (tail stream)
  (force (cdr stream)))%
\end{schemenoindent}

\bigskip
下記に示すのは減衰振動をモデル化した系
$$ C {dv_C \over dt} = -i_L - {v_C \over R}$$\nobreak
$$ L {di_L \over dt} = v_C$$
の積分に {\cf integrate-system} を応用した例である。

\begin{schemenoindent}
(define (damped-oscillator R L C)
  (lambda (state)
    (let ((Vc (vector-ref state 0))
          (Il (vector-ref state 1)))
      (vector (- 0 (+ (/ Vc (* R C)) (/ Il C)))
              (/ Vc L)))))

(define the-states
  (integrate-system
     (damped-oscillator 10000 1000 .001)
     '\#(1 0)
     .01))%
\end{schemenoindent}

