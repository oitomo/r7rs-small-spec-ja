\chapter{標準ライブラリ}
\label{stdlibraries}

%% Note, this is used to generate stdmod.tex.  The bindings could be
%% extracted automatically from the document, but this lets us choose
%% the ordering and optionally format manually where needed.

本節では,標準ライブラリで提供されるエクスポートを一覧表示する。
ライブラリは,すべての実装でサポートされないかもしれない,
あるいは負荷が高いかもしれない機能を分離するように考慮されている。

{\cf scheme} ライブラリ接頭辞にはすべての標準ライブラリで使用され,
将来の規格で使用するために予約されている。

\textbf{base ライブラリ}

\texttt{(scheme base)}ライブラリは,
伝統的に Scheme と関連付けられた多くの手続きと構文束縛をエクスポートする。
base ライブラリと他の標準ライブラリの境界は,構造上ではなく,使用上に基づく。
特に,他の標準手続きまたは構文の点で
コンパイラや実行システムによって通常プリミティブとして実装されている一部の手段は,
base ライブラリの一部としてではなく,分離したライブラリとして定義されている。
同様に, base ライブラリの一部のエクスポートは,他のエクスポートの点で実装可能である。
それらは厳密な意味で冗長ではあるが,
使い方の共通パターンを取り込んでいるので,便利な略語として提供されている。

\begin{scheme}
._               ...
.*                +                -
./                <=               <
.=>               =                >=
.>                abs              and
.append           apply            assoc
.assq             assv             begin
.boolean?          boolean=?       bytevector
.bytevector-copy  bytevector-append  bytevector-copy!
.bytevector-length bytevector-u8-ref bytevector-u8-set!
.bytevector?            caar             cadr
.call-with-current-continuation     call-with-values
.call-with-port
.call/cc          case
.car              cdr
.cdar    cddr
.ceiling          char->integer    char<=?
.char<?
.char=?           char>=?          char>?
.char?            complex?         cond
.cond-expand
.cons             define-syntax    define
.define-values
.define-record-type                 denominator
.do               dynamic-wind     else
.eq?
.equal?           eqv?             error
.error-object?    error-object-message  error-object-irritants
.even?            exact            exact-integer-sqrt
.exact-integer?   exact?           expt
.floor            for-each         gcd
.floor/     floor-quotient     floor-remainder
.truncate/  truncate-quotient  truncate-remainder
.features         guard            if
.include          include-ci
.inexact          inexact?
.integer->char    integer?         lambda
.lcm              length           let*
.let-syntax       letrec*          letrec-syntax
.let-values       let*-values
.letrec           let              list-copy
.list->string     list->vector     list-ref
.list-set!        list-tail        list?
.list             make-bytevector  make-list
.make-parameter   make-string      make-vector
.map              max              member
.memq             memv             min
.modulo           negative?        not
.null?            number->string   number?
.numerator        odd?             or
.pair?            parameterize
.positive?
.procedure?       quasiquote       quote
.quotient         raise-continuable
.raise            rational?        rationalize
.real?            remainder        reverse
.round            set!             set-car!
.set-cdr!         square
.string->list     string->number
.string->symbol   string->vector   string-append
.string-copy      string-copy!
.string-fill!     string-for-each
.string-length    string-map       string-ref
.string-set!      string<=?        string<?
.string=?         string>=?        string>?
.string?          string           substring
.symbol->string   symbol=?
.symbol?          syntax-error
.syntax-rules     truncate         values
.unquote          unquote-splicing
.vector-append    vector-copy      vector-copy!
.vector->list     vector->string   vector-fill!
.vector-for-each  vector-length    vector-map
.vector-ref       vector-set!      vector?
.vector           zero?            when
.with-exception-handler            unless
.binary-port?             char-ready?
.textual-port?            close-port
.close-input-port
.close-output-port        current-error-port
.current-input-port       current-output-port
.eof-object               eof-object?
.file-error?              flush-output-port
.get-output-string        get-output-bytevector
.input-port?              input-port-open?
.newline
.open-input-string        open-input-bytevector
.open-output-string       open-output-bytevector
.output-port?             output-port-open?
.peek-char
.peek-u8                  port?
.read-bytevector          read-bytevector!
.read-char                read-error?
.read-line                read-string
.read-u8                  string->utf8
.utf8->string             u8-ready?
.write-bytevector         write-char
.write-string             write-u8
\end{scheme}

\textbf{case-lambda ライブラリ}

\texttt{(scheme case-lambda)} ライブラリは,
{\cf case-lambda} 構文をエクスポートする。

\begin{scheme}
.case-lambda
\end{scheme}

\textbf{char ライブラリ}

\texttt{(scheme char)} ライブラリは,
すべてのUnicode文字をサポートする場合,潜在的に大きなテーブルを伴う
文字を扱う手続きを提供する。

\begin{scheme}
.char-alphabetic?
.char-ci=?       char-ci<?       char-ci>?
.char-ci<=?      char-ci>=?      char-upcase
.char-downcase   char-foldcase   char-lower-case?
.char-numeric?   char-upper-case?
.char-whitespace?                 string-ci=?
.string-ci<?     string-ci>?     string-ci<=?
.string-ci>=?    string-upcase   string-downcase
.string-foldcase
.digit-value
\end{scheme}

\textbf{complex ライブラリ}

\texttt{(scheme complex)} ライブラリは,
通常非実数のみ有用な手続きをエクスポートする。

\begin{scheme}
.angle   magnitude   imag-part   real-part
.make-polar           make-rectangular
\end{scheme}

\textbf{CxR ライブラリ}

\texttt{(scheme cxr)} ライブラリは,
3から4つの {\cf car} と {\cf cdr} 操作の合成からなる
24個の手続きをエクスポートする。

\begin{scheme}
(define caddar
  (lambda (x) (car (cdr (cdr (car x)))))){\rm.}%
\end{scheme}

{\cf car} と {\cf cdr} 自身,およびこれら2つの合成からなる4個の手続きは,
base ライブラリに含まれる。\ref{listsection}節参照。

\begin{scheme}
.caaaar caaadr caadar caaddr
.cadaar cadadr caddar cadddr
.cdaaar cdaadr cdadar cdaddr
.cddaar cddadr cdddar cddddr
.caaar caadr cadar caddr
.cdaar cdadr cddar cdddr
\end{scheme}

\textbf{eval ライブラリ}

\texttt{(scheme eval)} ライブラリは,
プログラムとして Scheme データを評価する手続きをエクスポートする。

\begin{scheme}
.eval
.environment
\end{scheme}

\textbf{file ライブラリ}

\texttt{(scheme file)} ライブラリは,
ファイルにアクセスするための手続きを提供する。

\begin{scheme}
.call-with-input-file    call-with-output-file
.delete-file             file-exists?
.open-input-file         open-output-file
.open-binary-input-file  open-binary-output-file
.with-input-from-file    with-output-to-file
\end{scheme}

\textbf{inexact ライブラリ}

\texttt{(scheme inexact)} ライブラリは,
通常不正確値のみ有用な手続きをエクスポートする。

\begin{scheme}
.acos      asin atan
.cos       exp  finite?
.infinite? log  nan?
.sin       sqrt tan
\end{scheme}

\textbf{lazy ライブラリ}

\texttt{(scheme lazy)} ライブラリは,
遅延評価のための手続きと構文キーワードをエクスポートする。

\begin{scheme}
.delay   delay-force   force   make-promise   promise?
\end{scheme}

\textbf{load ライブラリ}

\texttt{(scheme load)} ライブラリは,
Scheme 式をファイルからロードする手続きをエクスポートする。

\begin{scheme}
.load
\end{scheme}

\textbf{process-context ライブラリ}

\texttt{(scheme process-context)} ライブラリは,
プログラムの呼び出しコンテキストにアクセスする手続きをエクスポートする。

\begin{scheme}
.get-environment-variable
.get-environment-variables
.command-line
.emergency-exit
.exit
\end{scheme}

\textbf{read ライブラリ}

\texttt{(scheme read)} ライブラリは,
Scheme オブジェクトを読み込むための手続きを提供する。

\begin{scheme}
.read
\end{scheme}

\textbf{REPL ライブラリ}

\texttt{(scheme repl)} ライブラリは,
{\cf interaction-environment} 手続きをエクスポートする。

\begin{scheme}
.interaction-environment
\end{scheme}

\textbf{time ライブラリ}

\texttt{(scheme time)} ライブラリは,
時間に関連した値へのアクセスを提供する。

\begin{scheme}
.current-second
.current-jiffy
.jiffies-per-second
\end{scheme}

\textbf{write ライブラリ}

\texttt{(scheme write)} ライブラリは,
Scheme オブジェクトを書き込むための手続きを提供する。

\begin{scheme}
.write  write-shared write-simple  display
\end{scheme}

\textbf{R5RS ライブラリ}

\texttt{(scheme r5rs)} ライブラリは,
{\cf transcript-on} と {\cf transcript-off} が存在しないことを除いて,
\rfivers で定義された識別子を提供する。
{\cf exact} および {\cf inexact} 手続きは,
それらは \rfivers のもとでは,それぞれ
名前 {\cf inexact->exact} および {\cf exact->inexact} で現れる
ことに注意せよ。
しかし,もし実装が複素数ライブラリのような特定のライブラリを提供しなければ,
対応する識別子はこのライブラリに現れなくなる。

\begin{scheme}
.- * / + < <= = > >= abs acos and angle append apply asin assoc assq
.assv atan begin boolean?
.caaaar caaadr caadar caaddr
.cadaar cadadr caddar cadddr
.cdaaar cdaadr cdadar cdaddr
.cddaar cddadr cdddar cddddr
.caaar caadr cadar caddr
.cdaar cdadr cddar cdddr
.caar cadr cdar cddr
.call-with-current-continuation call-with-input-file call-with-output-file
.call-with-values car case cdr ceiling char->integer char-alphabetic?
.char-ci<? char-ci<=? char-ci=? char-ci>? char-ci>=? char-downcase
.char-lower-case? char-numeric? char-ready? char-upcase
.char-upper-case? char-whitespace? char? char<? char<=? char=? char>?
.char>=? close-input-port close-output-port complex? cond cons cos
.current-input-port current-output-port define define-syntax delay
.denominator display do dynamic-wind eof-object? eq? equal? eqv? eval
.even? exact->inexact exact? exp expt floor for-each force gcd if
.imag-part inexact->exact inexact? input-port? integer->char integer?
.interaction-environment lambda lcm length let let-syntax let* letrec
.letrec-syntax list list->string list->vector list-ref list-tail list?
.load log magnitude make-polar make-rectangular make-string make-vector
.map max member memq memv min modulo negative? newline not
.null-environment null? number->string number? numerator odd?
.open-input-file open-output-file or output-port? pair? peek-char
.positive? procedure? quasiquote quote quotient rational? rationalize
.read read-char real-part real? remainder reverse round
.scheme-report-environment set-car! set-cdr! set! sin sqrt string
.string->list string->number string->symbol string-append string-ci<?
.string-ci<=? string-ci=? string-ci>? string-ci>=? string-copy
.string-fill! string-length string-ref string-set! string? string<?
.string<=? string=? string>? string>=? substring symbol->string symbol?
.tan truncate values vector vector->list vector-fill! vector-length
.vector-ref vector-set! vector? with-input-from-file
.with-output-to-file write write-char zero?
\end{scheme}
